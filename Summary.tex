\chapterimage{sam.jpg}
\chapter{Summary}
\section{Mission Statement}\index{Mission Statement}

The mission of the Center for Simulation and Modeling is to foster multi-disciplinary research across the University of Pittsburgh to attack grand
challenge problems.  This is accomplished through

\begin{enumerate}
    \item enabling the application of high-performance computing to the
	  solution of problems through training, workshops, software, and
          hardware resources,
    \item providing environments and opportunities where researchers can
          communicate and collaborate,
    \item identifying and cultivating opportunities for synergistic activities.
\end{enumerate}


\section{Executive Summary}\index{Executive Summary}

The Center for Simulation and Modeling (SaM) has completed its seventh year of
operation.  This report covers the period June 1, 2014 through May 31, 2015.
Additionally, in order to be consistent with the University standard, the
listed publications are only those published in the calendar year 2014.

This has been an outstanding year for SaM.  The Center hosted its first HPC
Symposium and hosted Joost VandeVondele for a CP2K workshop, both of which
drew attendees from outside the University community.  The Consultants have
continued interacting with several research groups and published seven
peer-reviewed articles, while providing support for 621,012 jobs on the Frank
cluster utilizing nearly 11 million CPU hours.  The Center also welcomed a new
Executive Director.

After a significant expansion last year in the number of users from the School
of Medicine, we have seen far less growth this year in terms of the number of
new users.  However, the demand for storage at the scale of hundreds of
terabytes continues to grow.  Despite a recent expansion of our Lustre array,
demand for more capacity is high.  We are limited by our accounting structure,
which prevents us from collecting funds for storage purchases in ways that are
convenient for faculty.

As part of the educational activities of the Center, the SaM team led well-attended workshops on MPI-X parallel computing and Python programming and
gave three departmental seminars to increase the awareness of SaM among the
Pitt user community.  The collaboration between SaM and Computing Services and
Systems Development (CSSD) continues to be strengthened as the SaM team holds
biweekly meetings with CSSD personnel Louis Passarello, Jeff Raymond, and Jeff
White.  In addition, we helped set up and are continuing to oversee, jointly
with CSSD, the Frank computer cluster housed at the University's Network
Operations Center (NOC), at RIDC Park.  We have worked more closely with the NOC
this year to improve our management and networking infrastructure and have
several new systems on order to address critical infrastructure needs.  A new
monthly meeting including Louis Passarello and the Center Directors was
initiated to further enhance communication and planning.  Detailed information
on the activities in which the SaM team has been involved is provided below.


\section{Major Accomplishments (2014-2015)}\index{Major Accomplishments}

\begin{itemize}
    \item The Center hosted its first-ever HPC Symposium with Jack Wells of
          Oak Ridge National Laboratory serving as the keynote speaker.  This
          was very well-attended and included many attendees from outside the
          University community.

    \item Interactions with the School of Health Sciences continue to expand
	  and strengthen.  In particular, the storage systems required to
	  support genomics research have been upgraded and expanded.

    \item Center computer hardware and consulting services contributed to the
	  publication of 62 peer-reviewed journal publications in 2014.

\end{itemize}

