\chapterimage{sam.jpg}
\chapter{Infrastructure and Operations}\index{Infrastructure}\index{Operations}
The Center has continued our efforts to improve internal operations and our
support of major research efforts on campus.  Our SLASH2 filesystem,
implemented in collaboration with the Pittsburgh Supercomputing Center, continues
to be the primary storage system for the Pittsburgh Genome Resource Repository
(PGRR) and has necessitated networking changes to provide improved
functionality.

\section{SaM Team}\index{SaM Team}\index{Personnel}
The SaM Team has remained relatively unchanged this past year.  Antonio
Ferreira accepted the position of Executive Director with primary
responsibility for the day-to-day operations of the Center and Esteban Meneses
returned to Costa Rica at the end of June.  As of the writing of this report,
the Center is actively recruiting two new Consultants to replace Drs.\
Ferreira and Meneses.

In addition to the Consultants, SaM has received permission from the Provost
to recruit two new non-PhD positions: one to assist with systems
administration and the other to engage with the groups of Profs. Wilmer and Chong on scientific programming projects.  We expect these positions to be
filled by the Fall of 2015.

\begin{description}

    \item [Kenneth D. Jordan, Ph.D.] --- Co-Director: Richard King Mellon
	  Professor and Distinguished Professor of Computational Chemistry,
          Department of Chemistry.

    \item [J.\ Karl Johnson, Ph.D.] --- Co-Director: William Kepler Whiteford
          Professor, Department of Chemical and Petroleum Engineering.

    \item [M.\ Michael Barmada, Ph.D.] --- Associate Director: Associate
	  Professor, Departments of Human Genetics and Biomedical Informatics.

    \item [Antonio M.\ Ferreira, Ph.D.] --- Executive Director: Research
          Associate Professor, Departments of Chemistry and Computational \& Systems Biology.

    \item [Albert DeFusco, Ph.D.] --- Technical Director: Research
          Assistant Professor, Department of Chemistry.

    \item [Kim F.\ Wong, Ph.D.] --- Consultant: Research Assistant
          Professor, Department of Chemistry.

    \item [Esteban Meneses, Ph.D.] --- Consultant: Research Assistant
          Professor, Department of Chemistry

    \item [Patrick H.\ Pisciuneri, Ph.D.] --- Consultant: Research Assistant Professor, Department of Chemistry.

    \item [Senthil Natarajan] --- PittGrid Director. (CSSD staff
          member working in close collaboration with SaM consultants to
          provide grid computing resources on campus.)

    \item [Wendy Janocha] --- Administrative Coordinator.

\end{description}


\subsection{Personnel Development}\index{Personnel Development}
\subsubsection{Teaching}\index{Teaching}
\begin{itemize}
    \item Esteban taught a graduate-level course during Spring 2015 entitled CS1645/CS2405: Introduction to High Performance Computing Systems.

    \item Group developed and presented python workshop.
\end{itemize}



\subsubsection{Workshop Attendance}\index{Workshop Attendance}
\begin{itemize}
    \item Antonio Ferreira attended the \textit{Hands-on Workshop on
Computational Biophysics} (June 2015).

    \item Albert Defusco: NVIDIA GPU Tech conference (March 2015).

    \item Patrick H.\ Pisciuneri: Argonne Training Program on Extreme-Scale Computing (ATPESC), St.\ Charles, IL (August 2014).

\end{itemize}

\section{Hardware}\index{Hardware}
This year saw the first expansion to the Lustre filesystem, which now has a 480 TB usable capacity.  We also began improvements to our networking fabric to incorporate wider InfiniBand capability and build much needed redundancy into the fabric.  A new set of Haswell nodes were installed and will be available to users in the summer of 2015, with a new filesystem for user home directories on the way and a set of new nodes to support single-CPU jobs on order.

\section{Software}\index{Software}
Considerable effort was invested to evaluate new management tools for the SaM computational systems.  Several packages were considered, including the Platform LSF and Platform PCM (both from IBM), Bright Cluster Manager, Puppet, Ansible, and Warewulf.  Budgetary limitations ultimately limited the available choices and Warewulf was chosen as the cluster management tool going forward. The details of implementation and maintenance have been handed off to CSSD personnel at the NOC and we have begun using the new tools to manage the
Haswell nodes, which arrived in June 2015.  The open source tools Spack and
Lmod were chosen to facilitate the deployment of software packages on SaM
systems in an effort to further automate routine tasks of cluster
administration.  Lastly, SLURM was chosen as the new queue system and will be
presented to users beginning in late Summer of 2015.

Upgraded cluster support ticketing system.  Consultants can now update and
respond to tickets directly via e-mail and registered users can create a
new ticket simply by sending an e-mail to \texttt{help@sam.pitt.edu}.


\section{Allocations}\index{Allocations}
\subsection{New Groups and Users}\index{New Users}
\subsection{Service Unit Purchases}\index{Service Units, purchased}
Nine SU purchases totaling \$8,200 across eight departments.  This is
equivalent to 820,000 service units.
\begin{itemize}
    \item Andrew Daley (Physics and Astronomy): \$200
\end{itemize}

